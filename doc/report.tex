\documentclass{acm_proc_article-sp}
\usepackage{wrapfig}
\usepackage{tikz, alex, url, mu}
\usetikzlibrary{arrows,shapes,snakes,automata,backgrounds,petri}
\tikzset{>=stealth}

\title{Dynamic Cache Partitioning Based on Software Hints}

\author{Grant \and Jinilang}

\begin{document}
\maketitle

\begin{abstract}

\end{abstract}

\section{Introduction}

In traditional chip-multi-processors (CMPs), the last-level cache is 
shared among all cores and LRU is usually used as the replacement policy. Shared
last-level cache is long konwn to suffer from interference among cores. The 
LRU policy implicitly partitions the shared cache among cores based on their 
requests for cache resources. The resulted cache partitioning among cores is
unlikly to be optimal in terms of the total number of cache misses. This is 
because the marginal benefit gained from increasing cache space differ across 
cores. Limited cache resources should allocated to applications that may get
more marginal benefits from them.

Application developers might be able to predict their applications' cache usage 
and thus request an optimal amoumt of shared cache for his application. For 
example, the application developer might know that his application is a 
streaming application. In that case, although the application demands a large 
amount of cache, the marginal benefits the application gains from the additional
cache space is small. Thus he requests a small amount of cache according the 
application's working set size. So application programmers may provide hints for
requesting shared cache space.

Compiler might be able to analyze the program at compile time and predict the 
application's working set size and thus make request accordingly. The operating
sytem might be able to analyze the application's live trace and utilize more 
sophisticated algorithms to predict its working set size that what can be done
in hardware. Thus the operating system might as well provide hints for 
requesting shared cache space.

We understand that generating those hints might be a hard problem itself. 
However, as a logical first step, we'd like to know whether those hints would 
be helpful assuming idea hints can be generated. Therefore, in this project, we 
performed a limit study and designed a cache partitioning mechanism based on 
software hints to investigate how software hints may be utilized in cache 
partitioning to minimize cache misses. We believe our work can serve as the 
foundation of future works that intend to investigate cache partitioning based
on software hints.

\section{Background and Related Works}

Serveral cache partitioning mechanisms have been proposed by previous works 
to minimize cache misses in shared cache.
\cite{Qureshi:2006:UCP:1194816.1194855} proposed a utility-based cache 
partitioning strategy. In their approach, each core is associated with a utility
monitor (UMON). UMON has a tag directory which caches a tag per set per way. 
In order to reduce hardware overhead, UMON may cache only one tag for the same 
cache way in all sets. By using UMON, a cache miss line can be obtained, which 
depicts the relationship between number of cache misss and the number of ways 
assigned to this core. By greedily assigning each cache way to the core where 
the cache utility can be maximized (reducing most cache misses), this approach
partitions the cache to maximize utility (minimize cache misses). This is called
look ahead algorithm.

\cite{Qureshi:2006:CMC:1150019.1136501} pointed out that cache misses are not 
equivalently expensive. Parallel misses are much cheaper than isolated misses 
because they can be served in parallel. Instead of minimizing the number of 
cache misses, a better cache partitioning strategy might be minimizing MLP-based
cache cost. This strategy was explored by \cite{conf/IEEEpact/MoretoCRV07} which
demonstrated MLP-aware cache partitioning indeed achieves better performance.
Recently, \cite{conf/IEEEpact/BeckmannS13} studied partitioning fully-associated
cache to data and it reduced the cost of look-ahead algorithm by peek-ahead 
algorithm.

All previous works predict future cache usage based on previous cache miss 
curve. This works if the cache usage pattern remains similar across different 
phases of the program. When the memory usage pattern of the program changes 
across phases, the prediction might be inaccurate. In our project, we'd like to
further improve cache partitioning based on program hints for future memory 
usage given by users, OS, or compiler.

\cite{Ipek:2008:SMC:1381306.1382172} describes a novel memory controller design
 which would use adaptive 
scheduling based on machine learning. Using reinforcement learning, their 
scheduler would optimize scheduling on the fly. Controller-state action pairs 
are assigned reward values, and when commands are issued, the controller tries 
to choose the command with the greatest long term value. A learning controller 
brings about some great benefits to program performance. Primary amongst these 
is that the controller optimize for bus bandwidth, and does so on the fly. Many 
scheduling algorithms attempt provide the best bandwidth in the general case, 
though there often weak spots in their approaches which reduce memory 
throughput. By learning and being adaptable, the authors’ scheme fights this 
weakness. In addition, the rewards system takes the core where the memory 
request originated from, allowing the scheme to fight against starvation as 
well. However the capabilities of the scheduling system are limited. While in 
theory, the machine learning algorithm they chose should be able to take into 
account infinitely many states and inputs, hardware and computation time hampers
 the scheduling optimization possibilities. While the authors optimized their 
algorithm for the resources they had available, there are various scenarios 
which they were not able to account for due to hardware limitations, creating 
weak spots in their system.

\cite{10.1109/MM.2008.48} proposes a thread scheduling scheme to minimize LLC con-tention based on 
architectural observations made by the OS at run- time. While the paper does 
not discuss the modification of architectural features, it demonstrates how the 
OS can interface with the architecture to find optimize system behavior. In this
 particular instance, by leveraging features of the chosen architecture, the 
authors were able to track cache hit/miss ratios as well as absolute counts of 
hits and misses on caches in the system. Using these metrics, threads running in
 the system were assigned weights which corresponded to their cache usage. 
Threads were then scheduled on the cores in such a way that these weights were 
spread as evenly as possible across shared caches. The benefits to this approach
 are easily tangible. Scheduling threads in this way ensures that memory 
intensive processes are given an appropriate amount of resources rather than 
being forced to share an unnecessary and counter productive portion of cache. 
The feature we are proposing could possibly extend this scheme by allowing the 
OS to interface with shared caches to get better partitioning.

\bibliography{ref}
\bibliographystyle{abbrvnat}

\end{document}
